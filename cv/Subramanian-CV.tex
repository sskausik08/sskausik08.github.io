%% start of file `template.tex'.
%% Copyright 2006-2013 Xavier Danaux (xdanaux@gmail.com).
%
% This work may be distributed and/or modified under the
% conditions of the LaTeX Project Public License version 1.3c,
% available at http://www.latex-project.org/lppl/.


\documentclass[11pt,a4paper,sans]{moderncv}        % possible options include font size ('10pt', '11pt' and '12pt'), paper size ('a4paper', 'letterpaper', 'a5paper', 'legalpaper', 'executivepaper' and 'landscape') and font family ('sans' and 'roman')

% modern themes
\moderncvstyle{banking}                            % style options are 'casual' (default), 'classic', 'oldstyle' and 'banking'
\moderncvcolor{blue}                                % color options 'blue' (default), 'orange', 'green', 'red', 'purple', 'grey' and 'black'
%\renewcommand{\familydefault}{\sfdefault}         % to set the default font; use '\sfdefault' for the default sans serif font, '\rmdefault' for the default roman one, or any tex font name
%\nopagenumbers{}                                  % uncomment to suppress automatic page numbering for CVs longer than one page

% character encoding
\usepackage[utf8]{inputenc}                       % if you are not using xelatex ou lualatex, replace by the encoding you are using
%\usepackage{CJKutf8}                              % if you need to use CJK to typeset your resume in Chinese, Japanese or Korean

% adjust the page margins
\usepackage[scale=0.75]{geometry}
%\setlength{\hintscolumnwidth}{3cm}                % if you want to change the width of the column with the dates
%\setlength{\makecvtitlenamewidth}{10cm}           % for the 'classic' style, if you want to force the width allocated to your name and avoid line breaks. be careful though, the length is normally calculated to avoid any overlap with your personal info; use this at your own typographical risks...

\usepackage{import}

% personal data
\name{Kausik}{Subramanian}
%\title{Curriculum Vitae}                               % optional, remove / comment the line if not wanted
%\address{4 N Park Street, #407, Madison WI, 53715}{}{}% optional, remove / comment the line if not wanted; the "postcode city" and and "country" arguments can be omitted or provided empty
%\phone[mobile]{+1 608 960 5568}                   % optional, remove / comment the line if not wanted
%\phone[fixed]{01234 123456}                    % optional, remove / comment the line if not wanted
%\phone[fax]{+3~(456)~789~012}                      % optional, remove / comment the line if not wanted
\email{sskausik08@gmail.com / sskausik@google.com}                               % optional, remove / comment the line if not wanted
\homepage{sskausik08.github.io/}                         % optional, remove / comment the line if not wanted
%\extrainfo{additional information}                 % optional, remove / comment the line if not wanted
%\photo[64pt][0.4pt]{picture}                       % optional, remove / comment the line if not wanted; '64pt' is the height the picture must be resized to, 0.4pt is the thickness of the frame around it (put it to 0pt for no frame) and 'picture' is the name of the picture file
%\quote{Some quote}                                 % optional, remove / comment the line if not wanted

% to show numerical labels in the bibliography (default is to show no labels); only useful if you make citations in your resume
%\makeatletter
%\renewcommand*{\bibliographyitemlabel}{\@biblabel{\arabic{enumiv}}}
%\makeatother
%\renewcommand*{\bibliographyitemlabel}{[\arabic{enumiv}]}% CONSIDER REPLACING THE ABOVE BY THIS

% bibliography with mutiple entries
%\usepackage{multibib}
%\newcites{book,misc}{{Books},{Others}}
%----------------------------------------------------------------------------------
%            content
%----------------------------------------------------------------------------------
\begin{document}
%\begin{CJK*}{UTF8}{gbsn}                          % to typeset your resume in Chinese using CJK
%-----       resume       ---------------------------------------------------------
\makecvtitle

\vspace*{-21pt}
\small{My research lies in the application of formal reasoning and programming languages techniques for verification and synthesis of networks. I am currently a Software Engineer working on datacenter routing at Google.}

\section{Education}

\vspace{3pt}

\begin{itemize}
%\item{\cventry{}{M.S Computer Science  \hfill \emph{Fall 2017--Present}}{University of Wisconsin-Madison}{Madison}
%	{}{CGPA: 3.804/4.00}
%	{PhD Computer Science \hfill \emph{Fall 2017--Present}}\newline
%	{Advisors: Aditya Akella and Loris D'Antoni}}
%\vspace{3pt}
%\item{\cventry{2011--2015}{BTech. Computer Science and Engineering}{Indian Institute of Technology, Bombay}{Bombay}{}
%	{Advisors: Purushottam Kulkarni and Umesh Bellur}}

\item \textbf{University of Wisconsin-Madison} \\
\emph{PhD Computer Science}, CGPA: 3.868/4.00  \hfill \emph{Aug 2015 - July 2020} \\
Advisors: Aditya Akella and Loris D'Antoni
\vspace*{6pt}
\item \textbf{Indian Institute of Technology, Bombay} \\
\emph{BTech. Computer Science and Engineering} \hfill \emph{June 2011 - May 2015} \\
Advisors: Purushottam Kulkarni and Umesh Bellur

\end{itemize}

\section{Experience}
\begin{itemize}
	\item \textbf{Google, Sunnyvale, USA} \hfill \emph{Aug 2020 - Present} \\
	\emph{Software Engineer}, Routing Engine \hfill Manager: Leon Poutievski
	\begin{itemize}
		\item Working on Google's data center routing platform. Responsibilities 
		include ideation and implementation of new projects to 
		serve different customers of Routing Engine. 
	\end{itemize}
	\vspace*{6pt}

	% \emph{05/20/2019 - 08/09/2019 
	% \emph{11/04/2019 - 07/01/2020} 
	\item \textbf{Facebook Menlo Park, USA} \hfill \emph{May 2019 - Aug 2019} \\
	\emph{Software Engineering Intern} \hfill Mentors: Mahesh Maddikayala, Hyojeong Kim, James Zeng
	\begin{itemize}
		\item Worked on OpenR (https://github.com/facebook/openr), Facebook's internal routing platform. Implemented the
		Netlink protocol for OpenR ($\sim$3k LoC) to interface with the Linux kernel to program routes and listen to link/address/route events
		for protocol convergence.
	\end{itemize}
	\emph{Research Collaborator}, Network Routing \hfill \emph{Nov 2019 - Jul 2020} 
	\begin{itemize}
		\item Worked on presenting Facebook's operational experience of
		deploying and running an in-house BGP implementation in their data centers.
	\end{itemize}
	\vspace*{6pt}

	% \emph{05/15/2018 - 08/03/2018}
	\item \textbf{Microsoft Research, Cambridge, UK} \hfill \emph{May 2018 - Aug 2018} \\
	\emph{Research Intern}, Network Verification \hfill Mentors: Andrey Rybalchenko and Nuno Lopes
	\begin{itemize}
		\item Worked on developing a framework to for global MPLS tunnel path allocation for Microsoft's
		Wide Area Network. Using the framework, we analyse current production network allocation with
		the optimal allocation to make recommendations for improvements and future planning of the WAN.
	\end{itemize}
	\vspace*{6pt}

	%\emph{05/29/2017 - 09/01/2017}
	\item \textbf{Barefoot Networks, Santa Clara, USA} \hfill \emph{May 2017 - Sep 2017} \\
	\emph{Research Intern}, Advanced Applications \hfill Mentors: JK Lee, Robert Soule and Changhoon Kim
	\begin{itemize}
		\item Implemented various static analysis techniques for optimizing P4 programs in the Barefoot Tofino backend compiler
		pertaining to table dependencies and metadata usage based on P4
		developers' programming styles. Made several
		bug fixes to the \href{https://github.com/p4lang/p4c}{open-source P4 compiler}.
	\end{itemize}
	\vspace*{6pt}

	\item \textbf{Samsung Electronics, Suwon, South Korea} \hfill \emph{May 2014 - July 2014} \\
	\emph{Research Intern, Software R\&D Center} \hfill Mentors: Jeongshik In and Jaehoon Ko
	\begin{itemize}
		\item Proposed four Optimizations for Hadoop's Distributed File System. Analysed and
		modified HDFS to find the performance bottlenecks and add features to
		block placement and replication policy modules.
	\end{itemize}

	\vspace*{6pt}
	\item \textbf{Fraunhofer ITWM, Kaiserslautern, Germany} \hfill \emph{May 2013 - July 2013} \\
	\emph{Research Intern} \hfill Mentor: Mirko Rahn
		\begin{itemize}
			\item Implemented the Chord distributed hash table protocol using Fraunhofer's communication middleware GPI, which provides
			synchronous and asynchronous communication methods
		\end{itemize}

\end{itemize}

\vspace*{2pt}

\section{Patents \& Publications}
\begin{itemize}
	\item \textbf{DISTRIBUTED, PACKET-MEDIATED, PACKET ROUTING} \\
	Loris D’Antoni, Srinivsa Akella, \underline{Kausik Subramanian} \\
	\emph{US Patent No: US11075835B2}
	\begin{itemize}
		\item  A network switch holds a routing table and a network
		topology table so that when a link failure is detected at the
		network switch, the network switch may independently
		reroute a packet intended for that failed link using the
		network topology table. This processing can be performed in
		the data plane at a speed that can eliminate dropped packets.
	\end{itemize}
	\vspace*{4mm}

	\item \textbf{Doing more by doing less: how structured partial backpropagation improves deep learning clusters} \\
	 Adarsh Kumar, \underline{Kausik Subramanian}, Shivaram Venkataraman, and Aditya Akella \\
	\emph{DistributedML 2021: Proceedings of the 2nd ACM International Workshop on Distributed Machine Learning}
	\begin{itemize}
		\item  In this work, we exploit the unique characteristics of deep learning workloads to propose Structured Partial Backpropagation(SPB), a technique that systematically controls the amount of backpropagation at individual workers in distributed training. This simultaneously reduces network bandwidth, compute utilization, and memory footprint while preserving model quality.
	\end{itemize}
	\vspace*{4mm}

	\item \textbf{D2R: Policy-Compliant Fast Reroute} \\
	\underline{Kausik Subramanian}, Anubhavnidhi Abhashkumar, Loris D’Antoni, and Aditya Akella \\
	\emph{Proceedings of ACM SIGCOMM Symposium on SDN Research (SOSR) 2021}
	\begin{itemize}
		\item We take advantage of the recent advances in fast programmable switches 
		to perform policy-compliant route computations entirely in the data plane, 
		thus providing fast and programmable reactions to failures. D2R provides the 
		illusion of a hierarchical network fabric that is always available and 
		policy-compliant under failures. 
	\end{itemize}
	\vspace*{4mm}

	\item \textbf{Running BGP in Data Centers at Scale} \\
	Anubhavnidhi Abhashkumar*, Kausik Subramanian*, Alexey Andreyev, Hyojeong Kim,
	Nanda Kishore Salem, Jingyi Yang, Petr Lapukhov, Aditya Akella, and Hongyi Zeng\\
	\emph{Proceedings of 18th USENIX Symposium on Networked Systems Design and Implementation (NSDI '21)}
	\begin{itemize}
		\item We present Facebook’s BGP-based data center routing design and how it marries data center’s stringent
		requirements with BGP’s functionality, design and implementation of an in-house BGP stack and operational 
		experience of running BGP at scale.
		\item *Both authors contributed equally to this work.
	\end{itemize}
	\vspace*{4mm}
	
	\item \textbf{Detecting Network Load Violations for Distributed Control Planes} \\
	\underline{Kausik Subramanian}, Anubhavnidhi Abhashkumar, Loris D’Antoni, and Aditya Akella \\
	\emph{Proceedings of 41st ACM SIGPLAN Conference on Programming Language Design and Implementation (PLDI 2020), London, UK, (22\% acceptance rate)}
	\begin{itemize}
		\item By using an abstract representation of the control plane (ARC), we
		formulate a multi-node Mixed-Integer Linear Program which can be used to
		verify across machines if network links are overloaded (utilization
		exceeds capacity) under different failure scenarios. QARC models
		different routing protocols like OSPF and BGP and distributed load
		balancing strategies like ECMP/WCMP.
	\end{itemize}
	\vspace*{4mm}

	\item \textbf{Liveness Verification of Stateful Network Functions} \\
	Farnaz Yousefi, Anubhavnidhi Abhashkumar, \underline{Kausik Subramanian}, Kartik Hans, Soudeh Ghorbani, and Aditya Akella \\
	\emph{Conference: Proceedings of 17th USENIX Symposium on Networked Systems Design and Implementation (NSDI 2020), Santa Clara, California, USA, (18\% acceptance rate)}
	\begin{itemize}
		\item Liveness properties are important for stateful network function verification.
		In this work, we provide a compositional programming abstraction that decouples
		reachability from stateful network functions and model the behavior of the
		programs expressed in
		this abstraction using compact Boolean formulas.
		We provide a compiler that translates the
		programs written using our abstraction to P4 programs.
	\end{itemize}
	\vspace*{4mm}

	\item \textbf{Synthesis of Fault-Tolerant Distributed Router
		Configurations} \\
\underline{Kausik Subramanian}, Loris D’Antoni, and Aditya Akella \\
	\emph{Conference: Proceedings of the ACM on Measurement and Analysis of Computing Systems (SIGMETRICS 2018), Irvine, California, USA} \newline
	\emph{Journal: Proceedings of the ACM on Measurement and Analysis of Computing Systems Volume 2 Issue 1 March 2018 Article No.: 22 pp 1–26 (https://doi.org/10.1145/3179425)}
	\begin{itemize}
		\item A two phase synthesis algorithm for generating policy-compliant
		OSPF and BGP configurations which comply with high-level policies, even
		under failures. First, we use Genesis to synthesize a policy-compliant
		data plane, and then Zeppelin uses ILP solvers to generate OSPF and BGP
		configurations which converge to the policy-compliant data plane.
	\end{itemize}
	\vspace*{4mm}

	\item \textbf{Genesis: Synthesizing Forwarding Tables in Multi-tenant Networks} \\
	\underline{Kausik Subramanian}, Loris D’Antoni, and Aditya Akella \\
	\emph{Conference: 44th ACM SIGPLAN-SIGACT Symposium on Principles of Programming Languages (POPL 2017), Paris, France, (23\% acceptance rate)} \newline
	\emph{Journal: ACM SIGPLAN Notices Volume 52 Issue 1 January 2017 pp 572–585 (https://doi.org/10.1145/3093333.3009845)}
	\begin{itemize}
		\item A general and extensible approach to synthesize policy-compliant
		SDN forwarding tables for multi-tenant cloud settings using SMT solvers.
		Can support complex policies like reachability, waypoint traversal, path
		isolation and traffic engineering
	\end{itemize}
	\vspace*{4mm}
\end{itemize}

\section{Professional Service}
\begin{itemize}
\item Technical Program Committee - ACM SIGCOMM Symposium of SDN Research (SOSR) 2022
\item External Reviewer - ACM SIGCOMM 2021

\end{itemize}

\section{Talks and Posters}
\begin{itemize}
	\item D2R: Detecting Network Load Violations for Distributed Control Planes \\
	\emph{Talk at SOSR 2021 (Virtual)}

\item QARC: Detecting Network Load Violations for Distributed Control Planes \\
\emph{Talk at PLDI'20 (Virtual)}

\item Zeppelin: Synthesis of Fault-Tolerant Distributed Router Configurations \\
\emph{Talk at SIGMETRICS'18, Irvine, California, USA}

\item Genesis: Synthesizing Forwarding Tables in Multi-tenant Networks \\
\emph{Talk at POPL'17, Paris, France} \\
\emph{Talk at VMWare Research Group, August 2017}

\item Synthesizing Data and Control Planes for Multi-tenant Networks \\
\emph{Poster at Google Networking Research Summit 2017} \\
\emph{Poster at NSF workshop on Programmable Networks, NYU, 2018}
\end{itemize}
\section{Academic  Honors}
\begin{itemize}
\item{Awarded the UW-Madison CS Summer Research Assistantship, 2016.}
\item{Awarded Student Grants to attend SIGCOMM 2016, POPL 2017 and SIGMETRICS 2018.}
\item{Secured All India Rank 87 in IIT-JEE 2011 out of 485,000 students.}
\item{Secured All India Rank 3 in $10^{th}$ CBSE Board Examination, 2009. 
Invited by the Prime Minister's Office to witness the Republic Day 
Parade from the Prime Minister Box in New Delhi in 2010}
\end{itemize}

% \section{Technical and Personal skills}
% \begin{itemize}
% \item Proficient in Python, C++, Java, Z3, Gurobi, Latex, P4
% \item Familiar with Android, Hadoop, POX
% \end{itemize}

% \section{Courses}
% \begin{itemize}
%	\item \emph{Networks/Systems}: Advanced Networking, Big Data Systems
%	\item \emph{Programming Languages}: Program Verification and Synthesis,
%	Theory of Programming Languages, Advanced Compilation
%	\item \emph{Pedagogy}: Teaching in the College Classroom, Effective
%	Teaching in Internationally Diverse College Classroom
%	\item \emph{Miscellaneous}: Topics in Databases, Advanced Algorithms,
%	Computational Complexity Theory, Management and Marketing, and Accounting and Finance
%	for non-Business majors
%\end{itemize}

%\section{Positions of Responsibility}
%\begin{itemize}
%\item Mentor, Institute Student Mentorship Programme \hfill 2014-15 \\
%Mentoring a group of 12 freshmen and easing their transition to the academic %and social aspects of institute life. Also serving as a Department Academic %Mentor to 12 sophomores, guiding them about CS academic aspects.

%\item Internship Coordinator, Placement Cell \hfill 2013-14 \\
%Involved in the communication and scheduling of various companies as well as %universities and assisting them in the process of recruiting of students for %internships. Awarded Certificate of Appreciation by Dean, Academic Affairs %for exemplary work during the tenure.

%\end{itemize}


%% Publications from a BibTeX file without multibib
%%  for numerical labels: \renewcommand{\bibliographyitemlabel}{\@biblabel{\arabic{enumiv}}}% CONSIDER MERGING WITH PREAMBLE PART
%%  to redefine the heading string ("Publications"): \renewcommand{\refname}{Articles}
%\nocite{*}
%\bibliographystyle{plain}
%\bibliography{publications}                        % 'publications' is the name of a BibTeX file

% Publications from a BibTeX file using the multibib package
%\section{Publications}
%\nocitebook{book1,book2}
%\bibliographystylebook{plain}
%\bibliographybook{publications}                   % 'publications' is the name of a BibTeX file
%\nocitemisc{misc1,misc2,misc3}
%\bibliographystylemisc{plain}
%\bibliographymisc{publications}                   % 'publications' is the name of a BibTeX file

%-----       letter       ---------------------------------------------------------

\end{document}


%% end of file `template.tex'.
